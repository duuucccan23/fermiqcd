
\documentstyle[12pt,thmsa,sw20lart]{article}
%%%%%%%%%%%%%%%%%%%%%%%%%%%%%%%%%%%%%%%%%%%%%%%%%%%%%%%%%%%%%%%%%%%%%%%%%%%%%%%%%%%%%%%%%%%%%%%%%%%%%%%%%%%%%%%%%%%%%%%%%%%%
%TCIDATA{TCIstyle=article/art4.lat,lart,article}

%TCIDATA{Created=Sun Feb 13 20:08:19 2005}
%TCIDATA{LastRevised=Wed Feb 23 00:24:08 2005}

\input{tcilatex}
\begin{document}


\section{Introduction}

FermiQCD is a software library for fast development of parallel Lattice QCD
code. It includes examples and applications. The latest version of FermiQCD
supports:

\begin{itemize}
\item  Fully Object Oriented design

\item  Natural syntax

\item  Supports Wilson, Staggered and Domain-wall fermions

\item  Parallelization based on MPI is hidden to high level programmer

\item  SU3 operations optimized for P4 (SSE and SSE2 instructions)

\item  PSIM technology for emultaing parallel processes on a single
processor machine and improving performance on multi-threaded processors

\item  Compile and runs on multiple architectures including Linux, Mac and
Windows (with cygwin).
\end{itemize}

\subsection{Installation}

\begin{enumerate}
\item  Download the file fermiqcd\_4.0.tar.gz

\item  Execute: gunzip fermiqcd\_4.0.tar.gz

\item  Execute: tar xvf fermiqcd\_4.0.tar.gz
\end{enumerate}

\subsection{Files}

Upon installation fermiqcd creates the following directory structure
\begin{verbatim}
\FermiQCD
\FermiQCD\Version_4.0
\FermiQCD\Version_4.0\Libraries
\FermiQCD\Version_4.0\Documentation
\FermiQCD\Version_4.0\Examples
\FermiQCD\Version_4.0\Converters
\FermiQCD\Version_4.0\Tests
\FermiQCD\Version_4.0\Other
\end{verbatim}

The folder ``Libraries'' contains the software libraries. There is no need
to precompile anything since all the code is in the header files. This is
done to simplify usage and allow the compiler to perform better template
optimizations.

The files starting with {\tt mdp\_} belong to the Matrix Distributed
Processing (MDP) library, required by FermiQCD. The files starting with {\tt %
fermiqcd\_} are the proper FermiQCD files. They are distrubuted together but
are covered by different licenses.

The folder ``Documentation'' contains licenses and documentation for MDP and
FermiQCD. MDP and FermiQCD cannot be redistributed without this folder.

The folder ``Examples'' contains all of the examples described in these
tutorial.

The folder ``Converters'' contains converters for starndard QCD file formats
into the MDP file format used by FermiQCD.

You may ask: why another file format? Technically FermiQCD itself does not
specify a file format but it defines fields of objects defined on a lattice.
Field objects inherit parallel load/save operations from the underlying MDP
library which specifies a single file format for any generic field (any
number of lattice dimensions, any structure at the site) optimized for
parallel IO. None of the previous file formats was general enough since they
are all specific for a type of field and usually for 4 dimensions.

In any case FermiQCD can read UKQCD, MILC, CANOPY and many ASCII file
formats. If you format is not supported email me and I will send you a
converted in a week at no charge.

The folder ``Tests'' contains programs and libraries that I consider a work
in progress. They are included with the official distribution to allow
people to contribute.

The folder ``Other'' contains examples of applications in fields other than
QCD, for example Cellular Authomata.

\subsection{Compilation instructions}

To compile all the example go into /FermiQCD/Version\_4.0/Examples and type
\begin{verbatim}
make all
\end{verbatim}

or compile any of the programs with
\begin{verbatim}
g++ prg.cpp -I../Libraries -o prg.exe -O3 [options]
\end{verbatim}

You may want to edit the first few lines of the Makefile in order to enable
some specific compiler options:
\begin{verbatim}
g++ prg.cpp -I../Libraries -o prg.exe --O3 -DLINUX
\end{verbatim}

will compile for Linux and be able to measure running time
\begin{verbatim}
g++ prg.cpp -I../Libraries -o prg.exe -O3 -DUSE_DOUBLE_PRECISION
\end{verbatim}

will compile the algorithms to use double precision
\begin{verbatim}
g++ prg.cpp -I../Libraries -o prg.exe -O3 -DSSE2
\end{verbatim}

will compile the algorithms to use SSE/SSE2 optimizations (only on P4
processors)
\begin{verbatim}
g++ prg.cpp -I../Libraries -o prg.exe -O3 -DPARALLEL
\end{verbatim}

will compile with MPI, otherwise it will compile with PSIM Parallel
SIMulator (enables emulating parallel processing on a single processor PC,
it is recommended on a multi-threaded processor where it will increase
speed).

Compiler options can be combined.

\subsection{A first program}

\subsection{General principles}

More or less any FermiQCD the same structure:
\begin{verbatim}
#include ''fermiqcd.h''
int main(int argc, char** argv) {
   mdp.open_wormholes(argc,argv);
   // declare conventions
   // declare lattices
   // declare fields
   // declare variables
   // run algorithms
   mdp.close_wormholes();
   return 0;
}
\end{verbatim}

Where open\_wormholes and close\_wormholes respectively start and stop
communications.

Conventions are declared using the command
\begin{verbatim}
declare_base_matrices(''FERMIQCD'');
\end{verbatim}

which basically declares gamma matrices (Gamma[], Gamma5, and Gamma1). Other
conventions supported include ``UKQCD''.

A lattice, for examle a 4D lattice $16\times 8^3$ called {\tt mylattice} can
be declared as follows
\begin{verbatim}
int box[]={16,8,8,8};
mdp_lattice lattice(4,box);
\end{verbatim}

The constructor of the class can take optional parameters that are discussed
in the next section. The optional parameters specify how the lattice object
has to be partionioned over the parallel processes (default: by timeslices),
specify the lattice topology (default: mesh) and the size of the buffer used
for parallel communications (default: optimized for Wilson and Clover
actuions).

A {\bf lattice object} contains a parallel random number generator. To ask
every site to print a uniform random number:
\begin{verbatim}
#include ''fermiqcd.h''
int main(int argc, char** argv) {
   mdp.open_wormholes(argc,argv);
   declare_base_matrices(''FERMIQCD'');
   int box[]={16,8,8,8}
   mdp_lattice mylattice(4,box);
   site x(mylattice);
   forallsites(x)
      cout<<mylattice.random(x).plain()<<endl;
   mdp.close_wormholes();
   return 0;
}
\end{verbatim}

Class {\bf site} represents a site of the lattice; {\tt forallsites(x)} is a
parallel loop over all lattice sites using the site {\tt x} as looping
variable and
\begin{verbatim}
mylattice.random(x)
\end{verbatim}

is the {\bf random number generator} associated to the site {\tt x}. {\tt %
plain()} is a method of the random generator that returns a uniform random
number in (0,1). Other methods include
\begin{verbatim}
mylattice.random(x).SU(n)
\end{verbatim}

that generates a random SU(n) matrix using Cabibbo-Marinari.

To declare a {\bf field} of floating point numbers in single precision
called {\tt myfield} on our lattice
\begin{verbatim}
mdp_field<float> myfield(mylattice);
\end{verbatim}

To {\bf save} or {\bf load} a field\footnote{%
This functions may crash if you there is not enough memory to allocate
buffers for parallel IO. If this occurs pass a second {\tt int} argument to
the load/save functions with a value below 1024. The smoller the value, the
smaller the required buffer size.}
\begin{verbatim}
string filename;
myfield.save(filename);
myfield.load(filename);
\end{verbatim}

The following code creates a filed of floating point numbers, initialize
them at random and saves them
\begin{verbatim}
#include ''fermiqcd.h''
int main(int argc, char** argv) {
   mdp.open_wormholes(argc,argv);
   declare_base_matrices(''FERMIQCD'');
   int box[]={16,8,8,8}
   mdp_lattice mylattice(4,box);
   mdp_field<float> myfield(mylattice);
   site x(mylattice);
   forallsites(x)
      myfield(x)=mylattice.random(x).plain();
   myfield.update();
   myfield.save(''myfield.mdp'');
   mdp.close_wormholes();
   return 0;
}
\end{verbatim}

Note the function {\bf update()}. It is the single and most important
function in MDP and FermiQCD. It must be called every time a field changes
and before it is used. It instruct the parallel nodes to keep the copies of
the lattice sites and field variables therein sinchronized.

\subsection{Notes}

To loop over all sites of a given parity (EVEN or ODD)
\begin{verbatim}
forallsitesofparity(x,EVEN)
\end{verbatim}

To loop over all local sites and local copies of sites stored on other
processors
\begin{verbatim}
forallsitesandcopies(x)
\end{verbatim}

(This is used when looping to initialize a field with a local expression
that does not require the parallel random number generator, in order to
avoid a susequent call to the function update; if not sure, do not use it.)

To loop over all local even sites and local copies of even sites stored on
other processors:
\begin{verbatim}
forallsitesandcopiesofparity(x,EVEN)
\end{verbatim}

(Same as above.)

To print the time components of a site variable x
\begin{verbatim}
for(int k=0;k<x.lattice().ndim;k++)
   cout << x[k] << endl;
\end{verbatim}

Where {\tt x.lattice().ndim} reads as the number of dimensions ({\tt ndim})
of the lattice on which the site {\tt x} was declared. All fields and site
objects have a method lattice() to obtain a reference to the underlying
lattice. {\tt x[k]} reads as the {\tt k}-th coordinate of the site {\tt x}.
If the lattice was 4D x has four components 0,1,2, and 3.\ We adopt the
convention of calling coordinate 0 the TIME\ coordinate and 1,2,3 the space
coordinates.

\subsection{Other fields}

FermiQCD comes with a set of predefined fields
\begin{verbatim}
gauge_field
fermi_field
fermi_propagator
staggered_field
staggered_propagator
dw_fermi_field
dw_fermi_propagator
\end{verbatim}

and algorithms to create and used them. All the FermiQCD algorithms work
with any SU(n) gauge group although they are highly optimized for SU(3). The
staggered algorithms also work for any even-dimensional space. Some other
algorithms require a four dimensional space because of optimizations related
to the gamma matrix conventions.

\section{Matrix Distributed Processing}

\section{Short Lattice QCD tutorial and Notation}

\subsection{Modern Physics}

Modern Physics states that:

\begin{itemize}
\item  Any system can be uniquely descrived by a set of variables ($\psi
,A_\mu ,$ etc.) that represents its degrees of freedom and by a function of
these variables called Action that we indicate with the symbol $S_W(...)$.

\item  Any observable on the system can be associated to an expression,
function of the variables, that we indicate with $O(...)$.

\item  We cannot predict the outcome of every experiment but we can compute
the expectation value of any observable by computing the path integral 
\[
E[O(...)]=\int [d\psi ][dA_\mu ]O(...)e^{iS(...)}
\]
where the integral covers the domain of all possible states of the system
(i.e. one integral for each degree of freedom).

\item  Because information travels at a finite speed, $S(...)$ must be local
in space-time variables therefore it can be expressed as $S(...)=\int
L(...)d^4x$ where $L$ is called Lagrangian density and it is also function
of the degrees of freedom of the system.
\end{itemize}

\subsection{Quantum Chromo Dynamics}

If the system we are describing is QCD, i.e. the model that describes quarks
and gluons we have that:

\begin{itemize}
\item  $\psi _\alpha ^f(x)$ represent the degrees of freedom associated to
quarks.

\item  $A_\mu (x)$ represent the degrees of freedom associated to gluons

\item  The Action for QCD is 
\[
S_{QCD}(\psi ,A)=\int d^4x\left[ -\frac 1{4g^2}G_{\mu \nu }G^{\mu \nu }+%
\overline{\psi }_\alpha ^fD_f^{\alpha \beta }[A]\psi _\beta ^f\right] 
\]
where 
\[
G_{\mu \nu }=\partial _\mu A_\nu -\partial _\nu A_\mu +[A_\mu ,A_\nu ]
\]
\[
D_f^{\alpha \beta }[A]=\gamma _\mu ^{\alpha \beta }(\partial _\mu +A_\mu
)+1^{\alpha \beta }m_f
\]
Note that $A_\mu $ and $G_{\mu \nu }$ are $3\times 3$ complex matrices (i.e.
in adjoint representation) while $\psi _\alpha ^f$ for every $f$ and $\alpha 
$ is a $3$ components complex vector (i.e. fundamental representation).

\item  The most general path integral looks like 
\[
E[O(...)]=\int [d\psi ][dA_\mu ]O(...\psi ...A_\mu )P_{QCD}[\psi ,A]
\]
with 
\[
P_M[\psi ,A]=e^{i\int d^4x\left[ -\frac 1{4g^2}G_{\mu \nu }G^{\mu \nu }+%
\overline{\psi }_\alpha ^fD_f^{\alpha \beta }[A]\psi _\beta ^f\right] }
\]
\end{itemize}

\subsection{From QCD to Lattice QCD}

Lattice QCD provides a technique to define (i.e. regularize) the above
integral and compute the path integral analytically. This is achieved in
four steps:

\begin{enumerate}
\item  Perform a Wick rotation $x_0\rightarrow ix_0$ so that 
\[
P_{QCD}[\psi ,A]\rightarrow P[\psi ,A]=e^{-\int d^4x\left[ -\frac
1{4g^2}G_{\mu \nu }G_{\mu \nu }+\overline{\psi }_\alpha ^fD_f^{\alpha \beta
}[A]\psi _\beta ^f\right] }
\]
become real.

\item  Perform a Wick contraction of all spinors in $O(...)$ and replace 
\begin{eqnarray*}
...\overline{\psi }_\alpha ^f(x)...\psi _\beta ^f(y)... &\rightarrow
&S_{\alpha \beta }(A,x,y) \\
...\psi _\alpha ^f(x)...\overline{\psi }_\beta ^f(y)... &\rightarrow
&S_{\beta \alpha }(A,y,x)
\end{eqnarray*}
Note that $S=1/D[A]$ and it is called a propagator.

\item  Integrate out $[d\psi ]$ and obtain 
\[
E[O(...)]=\int [dA_\mu ]O(...S_{\alpha \beta }...A_\mu )P[A]
\]
\[
P[A]=e^{-\int d^4x\left[ -\frac 1{4g^2}G_{\mu \nu }G_{\mu \nu }\right]
-\sum_f\log \det D_f[A]}
\]

\item  Change variable from $A_\mu (x)$ to $U_\mu (x)=e^{iaA_\mu (x)}$ and
approximate the integral $[dA_\mu ]$ with a sum over $U_\mu ^{[k]}(x)$%
\[
E[O(...)]=\sum_kO(...S_{\alpha \beta }...U_\mu ^{[k]})P[U_\mu ^{[k]}]
\]
where 
\begin{eqnarray*}
P[U_\mu ^{[k]}] &=&e^{-S_{Gauge}-S_{Fermi}} \\
S_{Gauge} &=&\beta \sum_x\left[ \sum_{\mu \nu }P_{\mu \nu }(x)\right]  \\
S_{Fermi} &=&\sum_f\log \det D_f[U]
\end{eqnarray*}
$\beta =1/g^2$ is a regularized coupling 
\[
P_{\mu \nu }(x)=U_\mu (x)U_\nu (x+\mu )U_\mu ^H(x+\nu )U_\nu ^H(x)=-\frac{a^4%
}4G_{\mu \nu }G_{\mu \nu }
\]
is called a plaquette.
\end{enumerate}

$\beta =1/g^2$ is the only physical parameter of any lattice computation
other than quark masses. Because it sets the value of the regularized
coupling constant $g$, by virtue dimensional transmutation, it sets the
lattice scale in the computation.

Remember that computer programs deal with dimensionless number and not
dimensionful quantities. By chaning the value of $\beta $ one changes the
resolution of the computation. Typically a choice of $\beta =5.7$ is
equivalent to resolution of $1GeV^{-1}$. This relation is not linear and
depends on the discretization of the action.

\subsection{Lattice QCD computations}

Any Lattice QCD computation proceeds in the following steps:

\begin{itemize}
\item  Identify the expression $O(...)$, in terms of $S_{\alpha \beta }$ and 
$U$ that corresponds to the quantity to compute. In FermiQCD $S_{\alpha
\beta }$ and $U$ are implemented as C++ objects.

\item  Generate datasets $U_\mu ^{[k]}$ with probability given by $P[U_\mu
^{[k]}]$. This is achieved by a Markov Chain Monte Carlo and the datasets
(also called gauge configurations) are generated in succession starting from
a random one. In FermiQCD the markov chain is created by the {\tt heatbath}
algorithm.

\item  Evaluate the integrand $O(...)$ on each configuration and average the
results. Given a configuration $U$, $S_{\alpha \beta }$ is computed by
inverting $D_f^{\alpha \beta }[U]$ numerically. In FermiQCD this is done by
the {\tt generate} algorithm.

\item  Study the distribution of the results to estimate the statistical
error in the average. In FermiQCD this is done by the {\tt JackBoot} object.
\end{itemize}

Note that there are different ways to discretize the Action of QCD and they
affect the corvence rate of the numerical algorithm. FermiQCD include
various discretizations and it is easy to add more. There are also many
numerical techniques to invert $D_f^{\alpha \beta }[U]$ and compute $%
S_{\alpha \beta }$, FermiQCD implements the minimum residure and the
stabilized biconjugate gradient.

\subsection{On Fermions}

Often the term $S_{Fermi}$ is ignored in the computation. In a perturbative
language this is equivalent to ignore virtual quark loops. The effect of
this approximaiton, called quanching, is unclear but certainly
non-negligible.

Whether Fermions are treated dynamically or not one has to discretize $%
D_f^{\alpha \beta }[U]$ in order to invert it numerically. There are five
general prescriptions to do it:

\begin{itemize}
\item  Naive fermions. They do not work because lattice artifacts introduce
un-physical zeros in the propagator $S$.

\item  Wilson fermions. The most common and natural prescription. The only
problem is that it break the chiral symmetry and therefore Wilson fermions
cannot be masseless.

\item  Staggered (Kogut-Susskind) fermions. They do not break the chiral
symmetry completely and are more efficient than Wilson fermions but
introduce 15 additional quark flavors for each physical flavor. One can deal
with them but it is tricky.

\item  Domain-wall fermions. By introducing an unphysical extra-dimension
one can recover chirality and have almost massless quarks. They are
extreamly computationally expensive.

\item  Overlap (Neuberger) fermions. Have similar properties as Domain-wall
fermions but the extra dimension is integrated out analytically. They are
also very computationally expensive.
\end{itemize}

At this time FermiQCD suppors all the above types of fermions but the
Overlap. Domain-wall fermions have never been properly tested. Wilson and
Naive are implemented for both isotropic and un-isotropic lattices.
Staggered fermions are implemented for any even number of space-time
dimensions. All fermions and fermionic algorithms work for any SU(n) gauge
group.

\subsection{On the observables}

Because of the Wick rotation the numerical computation of expectation values
of observables $E[O(...)]$ that require an analytical continuation is
difficult. This makes difficult for example to compute scettering phases.

Nevertheless some observable can be parametrized in terms of quantity that
are not affected by the Wick rotation and can be computed numercally. They
include masses of composite particle and the modulus of most matrix elements.

Note that only gauge invariant quantities can be observable (remember the
Aaronov-Bohm effect). In the Lattice QCD language this means that any
observable $O(...)$ that is not gauge invariant will have a zero expectation
value. Non-trivial observables $O(...)$ are function of closed loops of
quark propagators $S$ and products of links $U$.

We can classify observables $O(...)$ in the following typical categories:

\begin{itemize}
\item  Pure glue that does not contain a length scale. They measure
topological properties of QCD vacuum.

\item  Pure glue that does contain a length scale such as the correlation
between two or more loops of links. They measure masses and matrix elements
of glueballs

\item  Product of two propagators $S$. They contain information about meson
masses.

\item  Product of three propagators $S$ contracted at the same two points.
They contain information about baryon masses.

\item  Other product of three or more propagators $S$ contracted in at least
three points. They contain information about matrix elements between mesons,
baryons or both.
\end{itemize}

Complete example programs in each category will be provided in the fourth
chapter.

\section{Quantum Chromo Dynamics}

\subsection{Pure gauge}

\subsubsection{Creating a hot gauge configuration}
\begin{verbatim}
#include ''fermiqcd.h''
int main(int argc, char** argv) {
   mdp.open_wormholes(argc,argv);
   declare_base_matrices(''FERMIQCD'');
   int box[]={16,8,8,8}
   mdp_lattice mylattice(4,box);
   int nc=3;
   gauge_field U(mylattice,nc);
   set_hot(U);
   U.save(''gauge.0000.mdp'');
   mdp.close_wormholes();
   return 0;
}
\end{verbatim}

the function set\_hot is already implemented but could have been implemented
easily as
\begin{verbatim}
void set_hot(gauge_field &U) {
   site x(U.lattice());
   forallsites(x)
      for(int mu=0; mu<U.ndim; mu++)
         U(x,mu)=U.lattice().random(x).SU(U.nc);
   U.update();
}
\end{verbatim}

\subsubsection{Creating a cold gauge configuration}
\begin{verbatim}
#include ''fermiqcd.h''
int main(int argc, char** argv) {
   mdp.open_wormholes(argc,argv);
   declare_base_matrices(''FERMIQCD'');
   int box[]={16,8,8,8}
   mdp_lattice mylattice(4,box);
   int nc=3;
   gauge_field U(mylattice,nc);
   set_cold(U);
   U.save(''gauge.0000.mdp'');
   mdp.close_wormholes();
   return 0;
}
\end{verbatim}

the function set\_hot is already implemented but could have been implemented
easily as
\begin{verbatim}
void set_cold(gauge_field &U) {
   site x(U.lattice());
   forallsites(x)
      for(int mu=0; mu<U.ndim; mu++)
         U(x,mu)=1;
   U.update();
}
\end{verbatim}

Note how each site initializes {\tt U(x,mu)} with 1 (interpreted as $3\times
3$ identity matrix). Since every site variable is initialized with a
constant and it does not depend on the random number generator it would be
more efficient to ask each process to initialize also the local copies of
remote site variables and avoid calling update
\begin{verbatim}
void set_cold(gauge_field &U) {
   site x(U.lattice());
   forallsitesandcopies(x)
      for(int mu=0; mu<U.ndim; mu++)
         U(x,mu)=1;
}
\end{verbatim}

This is how set\_cold is implemented in practice and it requires no parallel
communication.

\subsubsection{Performing Wilson heatbath steps}

Code to generate 100 gauge configurations $U_\mu ^{[k]}(x)$ and save them.
\begin{verbatim}
#include ''fermiqcd.h''
int main(int argc, char** argv) {
   mdp.open_wormholes(argc,argv);
   declare_base_matrices(''FERMIQCD'');
   int box[]={16,8,8,8}
   mdp_lattice mylattice(4,box);
   int nc=3;
   gauge_field U(mylattice,nc);
   set_cold(U);
   coefficients gauge;
   gauge[''beta'']=6.0;
   int niter=10;
   for(int k=0; k<100; k++) {
      WilsonGaugeAction::heatbath(U,gauge,niter);
      U.save(string(''gauge.'')+tostring(k)+string(''.mdp''));
   }
   mdp.close_wormholes();
   return 0;
}
\end{verbatim}

Note that gauge is a variable of type {\bf coefficients}, basically a hash
table that associates a floating point to any string. Variables of type
coefficients are used to pass parameters (or coefficients) to algorithms
that implement a physical action.

{\tt WilsonGaugeAction::heatbath} is the {\bf hetbath} algorithm using the
Wilson Gauge Action (who would have guessed?). It's first argument is the
gauge field is acts upon (reads and writes it), the second argument is the
set of coefficients (the only one it needs it ''beta'', i.e. the lattice
spacing), the third argument is the number of iterations before returning.
Noe that the number of interations technically is not a coefficient of the
action therefore it not passed as a variable in the gauge object.

The line
\begin{verbatim}
string(''gauge.'')+tostring(k)+string(''.mdp'')
\end{verbatim}

simply builds a filename without messing around with pointers since they are
unsafe.

\subsubsection{Improved actions}

All actions in FermiQCD are implemented as pure static classes (i.e. classes
with no member variables and only static methods). All gauge actions have a
static method heatbath that mush follow the same prototype.

For the Wilson gauge action
\begin{verbatim}
coefficients gauge;
gauge[''beta'']=6.0;    
WilsonGaugeAction::heatbath(U,gauge,niter);
\end{verbatim}

For the MILC improved gauge action
\begin{verbatim}
coefficients gauge;
gauge[''beta'']=6.0;    
gauge[''zeta'']=1.0;    
gauge[''u_t'']=1.0;    
gauge[''u_s'']=1.0;    
string model=''MILC'';
ImprovedGaugeAction::heatbath(U,gauge,niter,model);
\end{verbatim}

For the Morningstar improved gauge action
\begin{verbatim}
coefficients gauge;
gauge[''beta'']=6.0;    
gauge[''zeta'']=1.0;    
gauge[''u_t'']=1.0;    
gauge[''u_s'']=1.0;    
string model=''Morningstar'';
ImprovedGaugeAction::heatbath(U,gauge,niter,model);
\end{verbatim}

\subsubsection{Average plaquette}

Code to compute 
\[
\frac 16\frac 1{N_V}\frac 1{N_c}\func{Re}Tr\sum_{x,\mu >\nu }P_{\mu \nu }(x) 
\]
where $N_V$ is the lattice volume and $N_c$ is the number of colors.
\begin{verbatim}
#include ''fermiqcd.h''
int main(int argc, char** argv) {
   mdp.open_wormholes(argc,argv);
   declare_base_matrices(''FERMIQCD'');
   int box[]={16,8,8,8}
   mdp_lattice mylattice(4,box);
   int nc=3;
   gauge_field U(mylattice,nc);
   set_cold(U);
   coefficients gauge;
   gauge[''beta'']=6.0;
   int niter=10;
   for(int k=0; k<100; k++) {
      WilsonGaugeAction::heatbath(U,gauge,niter);
      U.save(string(''gauge.'')+tostring(k)+string(''.mdp''));
      mdp << average_plaquette(U) << endl;
   }
   mdp.close_wormholes();
   return 0;
}
\end{verbatim}

The function average\_plaquette computes the average plaquette on the gauge
field U. The output is a float number. Sending the output to mdp rather than
cout makes sure only process 0 prints the average plaquette even if all
processes contribute to the computation.

There is a similar function
\begin{verbatim}
average_plaquette(U,mu,nu)
\end{verbatim}

that computes the average plaquette considering the mu-nu plane only where
mu and nu are integers 
\[
\frac 1{N_V}\frac 1{N_c}\func{Re}Tr\sum_xP_{\mu \nu }(x) 
\]

The function average plaquette is already implemented as
\begin{verbatim}
myreal average_plaquette(gauge_field &U,int mu,int nu) {
 myreal tmp=0;
 site x(U.lattice());
 forallsites(x)
   tmp+=real(trace(plaquette(U,x,mu,nu)));
 mdp.add(tmp);
 return tmp/(U.lattice().nvol_gl*U.nc);
}
\end{verbatim}

where {\tt plaquette(U,x,mu,nu)} is defined as
\begin{verbatim}
U(x,mu)*U(x+mu,nu)*hermitian(U(x,nu)*U(x+nu,mu));
\end{verbatim}

mdp.add(tmp) adds the values tmp computed by the parallel processes (always
to be called after summing a variable inside a forallsites loop), and {\tt %
U.lattice().nvol\_gl} reads as the total number of lattice sites ({\tt %
nvol\_gl}) in the lattice for {\tt U}.

\subsubsection{Average path}

Note how the most general gluonic gauge invariant observable has the form 
\[
\oint_Ce^{iA_\mu dx^\mu } 
\]
where $C$ is a generic path. Here is the code to compute

\[
\frac 1{N_V}\frac 1{N_c}\func{Re}Tr\sum_x\oint_Ce^{iA_\mu dx^\mu } 
\]
\begin{verbatim}
#include ''fermiqcd.h''
int main(int argc, char** argv) {
   mdp.open_wormholes(argc,argv);
   declare_base_matrices(''FERMIQCD'');
   int box[]={16,8,8,8}
   mdp_lattice mylattice(4,box);
   int nc=3;
   gauge_field U(mylattice,nc);
   set_cold(U);
   coefficients gauge;
   gauge[''beta'']=6.0;
   int niter=10;
   int mu=0, nu=1;
   int path[6][2]={{+1,mu},{+1,mu},{+1,nu},{-1,mu},{-1,nu},{-1,nu}};
   for(int k=0; k<100; k++) {
      WilsonGaugeAction::heatbath(U,gauge,niter);
      U.save(string(''gauge.'')+tostring(k)+string(''.mdp''));
      mdp << average_path(U,6,path) << endl;
   }
   mdp.close_wormholes();
   return 0;
}
\end{verbatim}

The function {\tt average\_path} is a computes the average path on the gauge
field U. where the path $C$ is a spacified by a 2D array {\tt d} of links
and each each link is a verse (+1 or -1) and a direction (mu,
nu=0,1,2,3,...). Note that because a path may be highly non-local the
implementation of the funciton average\_path requires, in general, quite
some communication.

This function is already implemented and looks like the following:
\begin{verbatim}
mdp_complex average_path(gauge_field &U, int length, int d[][2]) {
   mdp_matrix_field psi(U.lattice(),U.nc,U.nc);
   mdp_site x(U.lattice());
   mdp_complex sum=0;
   for(int i=0; i<length; i++) {
     if(i==0) forallsites(x) psi(x)=U(x,d[i][0],d[i][1]);
     else forallsites(x) psi(x)*=U(x,d[i][0],d[i][1]);
     if(i<length-1) psi.shift(d[i][0],d[i][1]);
     else forallsites(x) sum+=trace(psi(x));
   }
   return sum/(U.lattice().nvol_gl*U.nc);
}
\end{verbatim}

Note tha call to the shift operator defined so that
\begin{verbatim}
psi.shift(+1,mu)
\end{verbatim}

assignes $\psi (x-\mu )$ to $\psi (x)$.

\subsection{Chromo-electro-magnetic field}

The Chromo-electro-magnetic field $P_{\mu \nu }=a^2G_{\mu \nu }$ has its own
class as a field of vectors of matrices but one never really needs to
declare it since it is uniquely associated to a aguge field. Given a gauge
field {\tt U} just call:
\begin{verbatim}
compute_em_field(U);
\end{verbatim}

and to obtain the plaquette $P_{\mu \nu }(x)$ just call
\begin{verbatim}
U.em(x,mu,nu)
\end{verbatim}

Therefore the average plaquette could also have been computes as
\begin{verbatim}
mdp_real sum=0.0;
compute_em_field(U);
forallsites(x)
   for(int mu=0; mu<U.ndim; mu++)
      for(int nu=mu+1; nu<U.ndim; nu++)
         sum+=real(trace(U.em(x,mu,nu));
retur sum/(U.lattice().nvol_gl*U.ndim*(U.ndim-1)/2*U.nc);
\end{verbatim}

\subsection{Fermions and inverters}

\subsection{Wilson fermions}

\subsection{Wilson mesons}

\subsection{Staggered fermions}

\subsection{Staggered mesons}

\subsection{Domain-wall fermions}

\subsection{Domain-wall mesons}

\subsection{Gauge fixing}

\subsection{Smearing}

\subsection{All-to-all propagators}

\subsection{Converting file formats}

\subsection{Implementing a new type of field}

\subsection{Implementing a new actions}

\end{document}
